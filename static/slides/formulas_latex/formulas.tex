\documentclass[12pt]{article}

\usepackage[brazil]{babel}
\usepackage[utf8]{inputenc}
\usepackage[T1]{fontenc}

\usepackage{graphicx, xcolor}
\usepackage{geometry}
\usepackage{amsmath, xfrac}
\usepackage{amssymb, bm}
\usepackage{lifecon}
\usepackage{fullpage}


% Alterando a fonte
\renewcommand{\familydefault}{\rmdefault}
\renewcommand*\rmdefault{ppl}

\title{\normalsize \bfseries Principais Fórmulas}
\date{\vspace{-2cm}}

\begin{document}

\maketitle


% Aula 12 - Matemática Financeira

$$a_{\lcroof{n}} = \frac{1-(1+i)^{-n}}{i} = \frac{1-v^n}{i} $$

$$\ddot{a}_{\lcroof{n}} = \frac{(1+i) .\, \left[ 1-(1+i)^{-n} \right]}{i} = \frac{1-v^n}{d} $$

$$a_{\lcroof{\infty}} = \frac{1+i}{i} = \frac{1}{d} $$

$$\ddot{a}_{\lcroof{\infty}} = \frac{1}{i} $$

$$s_{\lcroof{n}} = (1+i)^n .\, a_{\lcroof{n}} = \frac{(1+i)^n - 1}{i} $$


% Aula 13 - Tabelas de Vida

{\color{black!60}
$$e_x = \sum_{t=1}^\infty {}_{t}p_x $$ }

{\color{black!60}
$$e_{x:\lcroof{n}} = \sum_{t=1}^n {}_{t}p_x $$ }

{\color{black!60}
$$\mathring{e}_x = \int_{0}^\infty {}_{t}p_x \, dt$$ }

{\color{black!60}
$$\mathring{e}_{x:\lcroof{n}} = \int_{0}^n {}_{t}p_x \, dt$$ }

{\color{black!60}
$$\mathring{e}_{80:\lcroof{10}} = \int_{0}^{10} {}_{t}p_{80} \, dt$$ }






\end{document}