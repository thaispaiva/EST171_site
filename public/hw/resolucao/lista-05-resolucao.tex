\documentclass[]{article}
\usepackage{lmodern}
\usepackage{amssymb,amsmath}
\usepackage{ifxetex,ifluatex}
\usepackage{fixltx2e} % provides \textsubscript
\ifnum 0\ifxetex 1\fi\ifluatex 1\fi=0 % if pdftex
  \usepackage[T1]{fontenc}
  \usepackage[utf8]{inputenc}
\else % if luatex or xelatex
  \ifxetex
    \usepackage{mathspec}
  \else
    \usepackage{fontspec}
  \fi
  \defaultfontfeatures{Ligatures=TeX,Scale=MatchLowercase}
\fi
% use upquote if available, for straight quotes in verbatim environments
\IfFileExists{upquote.sty}{\usepackage{upquote}}{}
% use microtype if available
\IfFileExists{microtype.sty}{%
\usepackage{microtype}
\UseMicrotypeSet[protrusion]{basicmath} % disable protrusion for tt fonts
}{}
\usepackage[margin=1in]{geometry}
\usepackage{hyperref}
\hypersetup{unicode=true,
            pdftitle={Lista de Exercícios 5},
            pdfauthor={Thaís Paiva},
            pdfborder={0 0 0},
            breaklinks=true}
\urlstyle{same}  % don't use monospace font for urls
\usepackage{color}
\usepackage{fancyvrb}
\newcommand{\VerbBar}{|}
\newcommand{\VERB}{\Verb[commandchars=\\\{\}]}
\DefineVerbatimEnvironment{Highlighting}{Verbatim}{commandchars=\\\{\}}
% Add ',fontsize=\small' for more characters per line
\usepackage{framed}
\definecolor{shadecolor}{RGB}{248,248,248}
\newenvironment{Shaded}{\begin{snugshade}}{\end{snugshade}}
\newcommand{\AlertTok}[1]{\textcolor[rgb]{0.94,0.16,0.16}{#1}}
\newcommand{\AnnotationTok}[1]{\textcolor[rgb]{0.56,0.35,0.01}{\textbf{\textit{#1}}}}
\newcommand{\AttributeTok}[1]{\textcolor[rgb]{0.77,0.63,0.00}{#1}}
\newcommand{\BaseNTok}[1]{\textcolor[rgb]{0.00,0.00,0.81}{#1}}
\newcommand{\BuiltInTok}[1]{#1}
\newcommand{\CharTok}[1]{\textcolor[rgb]{0.31,0.60,0.02}{#1}}
\newcommand{\CommentTok}[1]{\textcolor[rgb]{0.56,0.35,0.01}{\textit{#1}}}
\newcommand{\CommentVarTok}[1]{\textcolor[rgb]{0.56,0.35,0.01}{\textbf{\textit{#1}}}}
\newcommand{\ConstantTok}[1]{\textcolor[rgb]{0.00,0.00,0.00}{#1}}
\newcommand{\ControlFlowTok}[1]{\textcolor[rgb]{0.13,0.29,0.53}{\textbf{#1}}}
\newcommand{\DataTypeTok}[1]{\textcolor[rgb]{0.13,0.29,0.53}{#1}}
\newcommand{\DecValTok}[1]{\textcolor[rgb]{0.00,0.00,0.81}{#1}}
\newcommand{\DocumentationTok}[1]{\textcolor[rgb]{0.56,0.35,0.01}{\textbf{\textit{#1}}}}
\newcommand{\ErrorTok}[1]{\textcolor[rgb]{0.64,0.00,0.00}{\textbf{#1}}}
\newcommand{\ExtensionTok}[1]{#1}
\newcommand{\FloatTok}[1]{\textcolor[rgb]{0.00,0.00,0.81}{#1}}
\newcommand{\FunctionTok}[1]{\textcolor[rgb]{0.00,0.00,0.00}{#1}}
\newcommand{\ImportTok}[1]{#1}
\newcommand{\InformationTok}[1]{\textcolor[rgb]{0.56,0.35,0.01}{\textbf{\textit{#1}}}}
\newcommand{\KeywordTok}[1]{\textcolor[rgb]{0.13,0.29,0.53}{\textbf{#1}}}
\newcommand{\NormalTok}[1]{#1}
\newcommand{\OperatorTok}[1]{\textcolor[rgb]{0.81,0.36,0.00}{\textbf{#1}}}
\newcommand{\OtherTok}[1]{\textcolor[rgb]{0.56,0.35,0.01}{#1}}
\newcommand{\PreprocessorTok}[1]{\textcolor[rgb]{0.56,0.35,0.01}{\textit{#1}}}
\newcommand{\RegionMarkerTok}[1]{#1}
\newcommand{\SpecialCharTok}[1]{\textcolor[rgb]{0.00,0.00,0.00}{#1}}
\newcommand{\SpecialStringTok}[1]{\textcolor[rgb]{0.31,0.60,0.02}{#1}}
\newcommand{\StringTok}[1]{\textcolor[rgb]{0.31,0.60,0.02}{#1}}
\newcommand{\VariableTok}[1]{\textcolor[rgb]{0.00,0.00,0.00}{#1}}
\newcommand{\VerbatimStringTok}[1]{\textcolor[rgb]{0.31,0.60,0.02}{#1}}
\newcommand{\WarningTok}[1]{\textcolor[rgb]{0.56,0.35,0.01}{\textbf{\textit{#1}}}}
\usepackage{graphicx,grffile}
\makeatletter
\def\maxwidth{\ifdim\Gin@nat@width>\linewidth\linewidth\else\Gin@nat@width\fi}
\def\maxheight{\ifdim\Gin@nat@height>\textheight\textheight\else\Gin@nat@height\fi}
\makeatother
% Scale images if necessary, so that they will not overflow the page
% margins by default, and it is still possible to overwrite the defaults
% using explicit options in \includegraphics[width, height, ...]{}
\setkeys{Gin}{width=\maxwidth,height=\maxheight,keepaspectratio}
\IfFileExists{parskip.sty}{%
\usepackage{parskip}
}{% else
\setlength{\parindent}{0pt}
\setlength{\parskip}{6pt plus 2pt minus 1pt}
}
\setlength{\emergencystretch}{3em}  % prevent overfull lines
\providecommand{\tightlist}{%
  \setlength{\itemsep}{0pt}\setlength{\parskip}{0pt}}
\setcounter{secnumdepth}{0}
% Redefines (sub)paragraphs to behave more like sections
\ifx\paragraph\undefined\else
\let\oldparagraph\paragraph
\renewcommand{\paragraph}[1]{\oldparagraph{#1}\mbox{}}
\fi
\ifx\subparagraph\undefined\else
\let\oldsubparagraph\subparagraph
\renewcommand{\subparagraph}[1]{\oldsubparagraph{#1}\mbox{}}
\fi

%%% Use protect on footnotes to avoid problems with footnotes in titles
\let\rmarkdownfootnote\footnote%
\def\footnote{\protect\rmarkdownfootnote}

%%% Change title format to be more compact
\usepackage{titling}

% Create subtitle command for use in maketitle
\newcommand{\subtitle}[1]{
  \posttitle{
    \begin{center}\large#1\end{center}
    }
}

\setlength{\droptitle}{-2em}
  \title{Lista de Exercícios 5}
  \pretitle{\vspace{\droptitle}\centering\huge}
  \posttitle{\par}
  \author{Thaís Paiva}
  \preauthor{\centering\large\emph}
  \postauthor{\par}
  \predate{\centering\large\emph}
  \postdate{\par}
  \date{13/05/2018}


\begin{document}
\maketitle

\hypertarget{matematica-financeira}{%
\subsection{Matemática Financeira}\label{matematica-financeira}}

\hypertarget{exercicio-1}{%
\subsubsection{Exercício 1}\label{exercicio-1}}

\begin{Shaded}
\begin{Highlighting}[]
\NormalTok{## taxa de juros efetiva}
\NormalTok{i =}\StringTok{ }\KeywordTok{nominal2Real}\NormalTok{(}\FloatTok{0.05}\NormalTok{,}\DecValTok{2}\NormalTok{)}
\NormalTok{v =}\StringTok{ }\DecValTok{1}\OperatorTok{/}\NormalTok{(}\DecValTok{1}\OperatorTok{+}\NormalTok{i)}

\NormalTok{VPbenef =}\StringTok{ }\DecValTok{1000}\OperatorTok{*}\NormalTok{v}\OperatorTok{^}\NormalTok{(}\DecValTok{10}\NormalTok{)}
\NormalTok{VPpgto =}\StringTok{ }\DecValTok{200} \OperatorTok{+}\StringTok{ }\DecValTok{500}\OperatorTok{*}\NormalTok{v}\OperatorTok{^}\DecValTok{6}

\NormalTok{x =}\StringTok{ }\NormalTok{(VPbenef }\OperatorTok{-}\StringTok{ }\NormalTok{VPpgto)}\OperatorTok{*}\NormalTok{(}\DecValTok{1}\OperatorTok{+}\NormalTok{i)}\OperatorTok{^}\DecValTok{15}
\end{Highlighting}
\end{Shaded}

Para que o valor presente dos fluxos de pagamentos seja zero, o último
pagamento deve ser de \$ 80.74.

\hypertarget{exercicio-2}{%
\subsubsection{Exercício 2}\label{exercicio-2}}

\begin{Shaded}
\begin{Highlighting}[]
\NormalTok{## função valor presente}
\NormalTok{VP =}\StringTok{ }\ControlFlowTok{function}\NormalTok{(x) }\DecValTok{500}\OperatorTok{*}\KeywordTok{annuity}\NormalTok{(}\DataTypeTok{i=}\NormalTok{x, }\DecValTok{15}\NormalTok{, }\DataTypeTok{m=}\DecValTok{5}\NormalTok{, }\DataTypeTok{type=}\StringTok{"due"}\NormalTok{) }\OperatorTok{-}\StringTok{ }\DecValTok{1000}\OperatorTok{*}\KeywordTok{annuity}\NormalTok{(}\DataTypeTok{i=}\NormalTok{x, }\DecValTok{5}\NormalTok{, }\DataTypeTok{type=}\StringTok{"due"}\NormalTok{)}

\NormalTok{## encontrando a raiz}
\NormalTok{TRI =}\StringTok{ }\KeywordTok{uniroot}\NormalTok{(VP, }\KeywordTok{c}\NormalTok{(}\DecValTok{0}\NormalTok{,}\DecValTok{1}\NormalTok{))}\OperatorTok{$}\NormalTok{root}
\end{Highlighting}
\end{Shaded}

A taxa interna de retorno é a taxa de juros que faz com que o valor
presente desse fluxo de pagamentos seja zero. Encontrando a raiz do VP
do fluxo de pagamentos, obtemos que a taxa interna de retorno é 4.29\%.

\hypertarget{exercicio-3}{%
\subsubsection{Exercício 3}\label{exercicio-3}}

\begin{Shaded}
\begin{Highlighting}[]
\NormalTok{p =}\StringTok{ }\DecValTok{37}\OperatorTok{*}\KeywordTok{annuity}\NormalTok{(}\DataTypeTok{i=}\FloatTok{0.08}\NormalTok{, }\DataTypeTok{n=}\OtherTok{Inf}\NormalTok{, }\DataTypeTok{type=}\StringTok{"due"}\NormalTok{) }\CommentTok{# perpetuidade imediata}
\end{Highlighting}
\end{Shaded}

Para contratar essa perpetuidade, seria apropriado pagar o seu valor
presente, igual a \$ 499.5.

\hypertarget{tabelas-de-vida}{%
\subsection{Tabelas de Vida}\label{tabelas-de-vida}}

\hypertarget{exercicio-4}{%
\subsubsection{Exercício 4}\label{exercicio-4}}

\begin{Shaded}
\begin{Highlighting}[]
\KeywordTok{data}\NormalTok{(demoChina)}

\NormalTok{## criando a tabela de vida}
\NormalTok{tbCL1 =}\StringTok{ }\KeywordTok{probs2lifetable}\NormalTok{(}\DataTypeTok{probs=}\NormalTok{demoChina}\OperatorTok{$}\NormalTok{CL1,}\DataTypeTok{radix=}\DecValTok{10000}\NormalTok{,}\DataTypeTok{type=}\StringTok{"qx"}\NormalTok{,}\StringTok{"CL1"}\NormalTok{)}
\KeywordTok{summary}\NormalTok{(tbCL1)}
\end{Highlighting}
\end{Shaded}

\begin{verbatim}
## This is lifetable:  CL1 
##  Omega age is:  105 
##  Expected curtated lifetime at birth is:  73.14131
\end{verbatim}

\hypertarget{exercicio-5}{%
\subsubsection{Exercício 5}\label{exercicio-5}}

\begin{Shaded}
\begin{Highlighting}[]
\NormalTok{p =}\StringTok{ }\KeywordTok{pxt}\NormalTok{(tbCL1, }\DataTypeTok{x=}\DecValTok{2}\NormalTok{, }\DataTypeTok{t=}\DecValTok{2}\NormalTok{)}
\end{Highlighting}
\end{Shaded}

De acordo com a tabela de vida em consideração, a probabilidade de uma
vida de 2 anos sobreviver até os 4 anos é 0.9971.

\hypertarget{exercicio-6}{%
\subsubsection{Exercício 6}\label{exercicio-6}}

\begin{Shaded}
\begin{Highlighting}[]
\NormalTok{d =}\StringTok{ }\NormalTok{tbCL1}\OperatorTok{@}\NormalTok{lx[tbCL1}\OperatorTok{@}\NormalTok{x}\OperatorTok{==}\DecValTok{35}\NormalTok{] }\OperatorTok{-}\StringTok{ }\NormalTok{tbCL1}\OperatorTok{@}\NormalTok{lx[tbCL1}\OperatorTok{@}\NormalTok{x}\OperatorTok{==}\DecValTok{45}\NormalTok{]}
\end{Highlighting}
\end{Shaded}

O número de mortes entre as idades 35 e 45 é dado por
\(l_{35} - l_{45}\), e é igual a 195.0743.

\hypertarget{exercicio-7}{%
\subsubsection{Exercício 7}\label{exercicio-7}}

\begin{Shaded}
\begin{Highlighting}[]
\NormalTok{p =}\StringTok{ }\KeywordTok{pxyzt}\NormalTok{(}\KeywordTok{list}\NormalTok{(tbCL1,tbCL1,tbCL1), }\KeywordTok{c}\NormalTok{(}\DecValTok{14}\NormalTok{,}\DecValTok{15}\NormalTok{,}\DecValTok{16}\NormalTok{), }\DecValTok{60}\NormalTok{)}
\end{Highlighting}
\end{Shaded}

A probabilidade de que três irmãos de idades 14, 15 e 16 estejam todos
vivos após 60 anos é 0.1777.

\hypertarget{matematica-atuarial}{%
\subsection{Matemática Atuarial}\label{matematica-atuarial}}

\hypertarget{exercicio-8}{%
\subsubsection{Exercício 8}\label{exercicio-8}}

\begin{Shaded}
\begin{Highlighting}[]
\NormalTok{## criar tabela atuarial}
\NormalTok{actCL1 =}\StringTok{ }\KeywordTok{new}\NormalTok{(}\StringTok{"actuarialtable"}\NormalTok{, }\DataTypeTok{x=}\NormalTok{tbCL1}\OperatorTok{@}\NormalTok{x, }\DataTypeTok{lx=}\NormalTok{tbCL1}\OperatorTok{@}\NormalTok{lx, }\DataTypeTok{interest=}\FloatTok{0.05}\NormalTok{)}

\NormalTok{## VPA seguro dotal puro}
\NormalTok{S =}\StringTok{ }\DecValTok{100000}
\NormalTok{VPA =}\StringTok{ }\NormalTok{S}\OperatorTok{*}\KeywordTok{Exn}\NormalTok{(actCL1, }\DataTypeTok{x=}\DecValTok{25}\NormalTok{, }\DataTypeTok{n=}\DecValTok{65-25}\NormalTok{)}
\end{Highlighting}
\end{Shaded}

Para receber \$ 100.000 daqui a 40 anos caso sobreviva, um segurado de
25 deverá pagar hoje \$ 11426.64.

\hypertarget{exercicio-9}{%
\subsubsection{Exercício 9}\label{exercicio-9}}

\begin{Shaded}
\begin{Highlighting}[]
\NormalTok{## VPA seguro dotal misto}
\NormalTok{VPA =}\StringTok{ }\NormalTok{S}\OperatorTok{*}\KeywordTok{AExn}\NormalTok{(actCL1, }\DataTypeTok{x=}\DecValTok{25}\NormalTok{, }\DataTypeTok{n=}\DecValTok{65-25}\NormalTok{)}
\end{Highlighting}
\end{Shaded}

Para receber \$ 100.000 daqui a 40 anos caso sobreviva, ou quando morrer
nesse período, um segurado de 25 deverá pagar hoje \$ 16498.73.

\hypertarget{exercicio-10}{%
\subsubsection{Exercício 10}\label{exercicio-10}}

\begin{Shaded}
\begin{Highlighting}[]
\NormalTok{## VPA seguro de vida inteira}
\NormalTok{VPA =}\StringTok{ }\NormalTok{S}\OperatorTok{*}\KeywordTok{Axn}\NormalTok{(actCL1, }\DataTypeTok{x=}\DecValTok{30}\NormalTok{, }\DataTypeTok{k=}\DecValTok{12}\NormalTok{, }\DataTypeTok{i=}\FloatTok{0.04}\NormalTok{)}
\end{Highlighting}
\end{Shaded}

O VPA de um seguro de vida inteira para um indivíduo de 30 anos com
indenização de \$ 100.000 paga no final do mês de morte e juros de 4\% é
\$ 19391.06.

\hypertarget{exercicio-11}{%
\subsubsection{Exercício 11}\label{exercicio-11}}

\begin{Shaded}
\begin{Highlighting}[]
\NormalTok{## prêmio mensal}
\NormalTok{P =}\StringTok{ }\NormalTok{(VPA}\OperatorTok{/}\KeywordTok{axn}\NormalTok{(actCL1, }\DataTypeTok{x=}\DecValTok{30}\NormalTok{, }\DataTypeTok{k=}\DecValTok{12}\NormalTok{, }\DataTypeTok{i=}\FloatTok{0.04}\NormalTok{))}\OperatorTok{/}\DecValTok{12}
\end{Highlighting}
\end{Shaded}

Se o seguro do exercício anterior for pago com prêmios mensais
vitalícios, o valor de cada prêmio será \$ 78.5.

\hypertarget{exercicio-12}{%
\subsubsection{Exercício 12}\label{exercicio-12}}

\begin{Shaded}
\begin{Highlighting}[]
\NormalTok{## prêmio trimestral}
\NormalTok{P =}\StringTok{ }\NormalTok{(S}\OperatorTok{*}\KeywordTok{Axn}\NormalTok{(actCL1, }\DataTypeTok{x=}\DecValTok{50}\NormalTok{, }\DataTypeTok{k=}\DecValTok{4}\NormalTok{)}\OperatorTok{/}\KeywordTok{axn}\NormalTok{(actCL1, }\DataTypeTok{x=}\DecValTok{50}\NormalTok{, }\DataTypeTok{k=}\DecValTok{4}\NormalTok{))}\OperatorTok{/}\DecValTok{4}
\end{Highlighting}
\end{Shaded}

O prêmio que um segurado de 50 anos deverá pagar trimestralmente para
contratar esse seguro vitalício, também trimestral, é de \$ 541.65.

\hypertarget{exercicio-13}{%
\subsubsection{Exercício 13}\label{exercicio-13}}

\begin{Shaded}
\begin{Highlighting}[]
\NormalTok{## prêmio mensal}
\NormalTok{P =}\StringTok{ }\NormalTok{(S}\OperatorTok{*}\KeywordTok{Axn}\NormalTok{(actCL1, }\DataTypeTok{x=}\DecValTok{30}\NormalTok{, }\DataTypeTok{n=}\DecValTok{35}\NormalTok{, }\DataTypeTok{k=}\DecValTok{12}\NormalTok{, }\DataTypeTok{i=}\FloatTok{0.03}\NormalTok{)}\OperatorTok{/}\KeywordTok{axn}\NormalTok{(actCL1, }\DataTypeTok{x=}\DecValTok{30}\NormalTok{, }\DataTypeTok{n=}\DecValTok{15}\NormalTok{, }\DataTypeTok{k=}\DecValTok{12}\NormalTok{, }\DataTypeTok{i=}\FloatTok{0.03}\NormalTok{))}\OperatorTok{/}\DecValTok{12}
\end{Highlighting}
\end{Shaded}

O prêmio mensal, pago durante 15 anos, por um segurado de 30 anos para
um seguro temporário por 35 anos é de \$ 65.07.


\end{document}
